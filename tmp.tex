\documentclass[english,man,doc,11pt, twoside,floatsintext]{apa6}
\usepackage{lmodern}
\usepackage{amssymb,amsmath}
\usepackage{ifxetex,ifluatex}
\usepackage{fixltx2e} % provides \textsubscript
\ifnum 0\ifxetex 1\fi\ifluatex 1\fi=0 % if pdftex
  \usepackage[T1]{fontenc}
  \usepackage[utf8]{inputenc}
\else % if luatex or xelatex
  \ifxetex
    \usepackage{mathspec}
  \else
    \usepackage{fontspec}
  \fi
  \defaultfontfeatures{Ligatures=TeX,Scale=MatchLowercase}
\fi
% use upquote if available, for straight quotes in verbatim environments
\IfFileExists{upquote.sty}{\usepackage{upquote}}{}
% use microtype if available
\IfFileExists{microtype.sty}{%
\usepackage{microtype}
\UseMicrotypeSet[protrusion]{basicmath} % disable protrusion for tt fonts
}{}
\usepackage{hyperref}
\hypersetup{unicode=true,
            pdftitle={Thesis proposal: Diffusion of innovation in sustainability in a context of Dutch energy transition},
            pdfauthor={Marco Bunt},
            pdfkeywords={Sustainability, Co-creation, diffusion of innovation, non-adopters},
            pdfborder={0 0 0},
            breaklinks=true}
\urlstyle{same}  % don't use monospace font for urls
\ifnum 0\ifxetex 1\fi\ifluatex 1\fi=0 % if pdftex
  \usepackage[shorthands=off,main=english]{babel}
\else
  \usepackage{polyglossia}
  \setmainlanguage[]{english}
\fi
\usepackage{graphicx,grffile}
\makeatletter
\def\maxwidth{\ifdim\Gin@nat@width>\linewidth\linewidth\else\Gin@nat@width\fi}
\def\maxheight{\ifdim\Gin@nat@height>\textheight\textheight\else\Gin@nat@height\fi}
\makeatother
% Scale images if necessary, so that they will not overflow the page
% margins by default, and it is still possible to overwrite the defaults
% using explicit options in \includegraphics[width, height, ...]{}
\setkeys{Gin}{width=\maxwidth,height=\maxheight,keepaspectratio}
\IfFileExists{parskip.sty}{%
\usepackage{parskip}
}{% else
\setlength{\parindent}{0pt}
\setlength{\parskip}{6pt plus 2pt minus 1pt}
}
\setlength{\emergencystretch}{3em}  % prevent overfull lines
\providecommand{\tightlist}{%
  \setlength{\itemsep}{0pt}\setlength{\parskip}{0pt}}
\setcounter{secnumdepth}{0}
% Redefines (sub)paragraphs to behave more like sections
\ifx\paragraph\undefined\else
\let\oldparagraph\paragraph
\renewcommand{\paragraph}[1]{\oldparagraph{#1}\mbox{}}
\fi
\ifx\subparagraph\undefined\else
\let\oldsubparagraph\subparagraph
\renewcommand{\subparagraph}[1]{\oldsubparagraph{#1}\mbox{}}
\fi

%%% Use protect on footnotes to avoid problems with footnotes in titles
\let\rmarkdownfootnote\footnote%
\def\footnote{\protect\rmarkdownfootnote}


  \title{Thesis proposal: Diffusion of innovation in sustainability in a context
of Dutch energy transition}
    \author{Marco Bunt\textsuperscript{1,2}}
    \date{}
  
\shorttitle{Diffusion of innovation in sustainability}
\affiliation{
\vspace{0.5cm}
\textsuperscript{1} Erasmus school of social and behavioural sciences\\\textsuperscript{2} Stedin netbeheer}
\keywords{Sustainability, Co-creation,  diffusion of innovation, non-adopters\newline\indent Word count: X}
\usepackage{csquotes}
\usepackage{upgreek}
\captionsetup{font=singlespacing,justification=justified}

\usepackage{longtable}
\usepackage{lscape}
\usepackage{multirow}
\usepackage{tabularx}
\usepackage[flushleft]{threeparttable}
\usepackage{threeparttablex}

\newenvironment{lltable}{\begin{landscape}\begin{center}\begin{ThreePartTable}}{\end{ThreePartTable}\end{center}\end{landscape}}

\makeatletter
\newcommand\LastLTentrywidth{1em}
\newlength\longtablewidth
\setlength{\longtablewidth}{1in}
\newcommand{\getlongtablewidth}{\begingroup \ifcsname LT@\roman{LT@tables}\endcsname \global\longtablewidth=0pt \renewcommand{\LT@entry}[2]{\global\advance\longtablewidth by ##2\relax\gdef\LastLTentrywidth{##2}}\@nameuse{LT@\roman{LT@tables}} \fi \endgroup}



\authornote{This article is the graduation thesis om Marco
Bunt for the study social science on the Erasmus school of social and
behavioral sciences, in collaboration with Stedin.

Correspondence concerning this article should be addressed to Marco
Bunt, Stedin, Blaak 8, 3011 TA Rotterdam. E-mail:
\href{mailto:marco.bunt@Stedin.net}{\nolinkurl{marco.bunt@Stedin.net}}}

\abstract{
Yet to be written


}

\begin{document}
\maketitle

\section{\texorpdfstring{Study 1: identify non-adapters
\label{Study1}}{Study 1: identify non-adapters }}\label{study-1-identify-non-adapters}

The first part of this study is about the identification (and
localization) of people that are not adopting ECO-innovation. This part
of the study is mainly conducted as a preparation for the
\hyperref[Study2]{second part} of this article.

\subsection{Method}\label{method}

\subsubsection{Data}\label{data}

(``Rotterdam in cijfers,'' n.d.) contains data about the city Rotterdam
and used as a source for the following data: residential mobility,
election results, annual income. (``Stedin,'' n.d.) provided the data
for the geographic location of PV. (``Basisregistratie adressen en
gebouwen (bag),'' n.d.) is used for information about the houses in the
area.

\subsubsection{Procedure}\label{procedure}

Insight in the variables predicting (non-)adoption are created by
creating odd-ratio tables and conducting a multivariable regression
analysis. The data is divided in separate groups: 1) \enquote{Natural
variables}, containing the natural part of the Co-production as
described by Jasanoff (2004)\footnote{Not sure is I can already} and the
part about the 2) \enquote{imaginaries}. The former variables are

\subsection{Results and Discussion}\label{results-and-discussion}

\begin{table}[t]

\caption{\label{tab:RegTab}parameters regression analysis}
\centering
\begin{tabular}{llll}
\toprule
Variable & Model\_1 & Model\_2 & Model\_3\\
\midrule
\addlinespace[0.3em]
\multicolumn{4}{l}{\textbf{Natural resources}}\\
\hspace{1em}InstalledPV & x &  & x\\
\hspace{1em}HousePrice & x &  & x\\
\hspace{1em}SizeHouse & x &  & x\\
\hspace{1em}ConstructYear & x &  & x\\
\hspace{1em}SizeRooftop & x &  & x\\
\addlinespace[0.3em]
\multicolumn{4}{l}{\textbf{Imaginaries }}\\
\hspace{1em}VerkiezingUitslag (1) &  & x & x\\
\addlinespace[0.3em]
\multicolumn{4}{l}{\textbf{contextual parameters}}\\
\hspace{1em}Ownership & x & x & x\\
\hspace{1em}DistrHeat & x & x & x\\
\hspace{1em}ResidentMobility & x & x & x\\
\hspace{1em}Income & x & x & x\\
\bottomrule
\multicolumn{4}{l}{\textsuperscript{a} I'm not sure abouth already investigate the}\\
\multicolumn{4}{l}{imagionaries in the first part}\\
\end{tabular}
\end{table}

\subsubsection{Thresholds and benefits}\label{thresholds-and-benefits}

\subsubsection{Diffusion of solar}\label{diffusion-of-solar}

\newpage

\section{References}\label{references}

\begingroup
\setlength{\parindent}{-0.5in} \setlength{\leftskip}{0.5in}

\hypertarget{refs}{}
\hypertarget{ref-BAG}{}
Basisregistratie adressen en gebouwen (bag). (n.d.).
\emph{Basisregistratie Adressen en Gebouwen (BAG)}. Retrieved from
\url{https://bag.basisregistraties.overheid.nl/}

\hypertarget{ref-jasanoff_2004}{}
Jasanoff, S. (2004). States of knowledge: The co-production of science
and social order.
doi:\href{https://doi.org/10.4324/9780203413845}{10.4324/9780203413845}

\hypertarget{ref-rotterdam_in_cijfers}{}
Rotterdam in cijfers. (n.d.). \emph{Rotterdam in Cijfers}. Retrieved
from \url{https://rotterdam.incijfers.nl/}

\hypertarget{ref-stedin}{}
Stedin. (n.d.). \emph{Thuis \textbar{} Stedin}. Retrieved from
\url{http://www.stedin.net/}

\endgroup


\end{document}
